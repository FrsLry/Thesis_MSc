%% Matériel et Méthode

\chapter{Matériel et Méthode}
    \section{Modélisation qualitative}
        \subsection{Description de l'approche qualitative}
La modélisation qualitative est une méthode développée par Puccia et Levins et présentée dans « Qualitative Modeling of Complex Systems » (1985), qui démontre la pertinence et les intérêts d’utiliser l’approche qualitative. Un modèle est dit qualitatif lorsque seul les signes des interactions, entre les variables d’états sont connus, indépendamment de leur magnitude (Puccia et Levins, 1991). Cette approche peut être utilisée pour plusieurs raisons : (1) les variables utilisées peuvent être à la fois qualitatives et quantitatives, (2) le manque de données qualitatives peut être contourné, (3) l’écosystème peut être étudié de manière holistique grâce à une complexité réduite et/ou (4) le modèle peut être construit directement à partir de la littérature scientifique et par dires d’experts. L’approche qualitative doit en revanche se soumettre à l’hypothèse forte que les différentes variables du modèle sont à l’état d’équilibre, i.e. que la variation des variables au cours du temps est égale à zéro (méthode de la moyenne temporelle). Grâce à cette approche, il est possible d’observer la réponse de chaque variable après la perturbation (i.e. augmentation ou diminution) d’une de ces dernières. Chaque perturbation de variable peut être considéré comme un scénario (Tab. n), pour lequel on simule l’augmentation ou la diminution de l’abondance d’une espèce ou d’une ressource. Il est aussi possible de simuler des scénarios plus complexes dans lesquels on observe les variations de plusieurs variables simultanément. Les perturbations simulées ici doivent être considérée comme constante, continue et prolongée dans le temps pour respecter la méthode de la moyenne temporelle (i.e. perturbation persistante).Le schéma conceptuel représenté dans la Fig.2) est un graphique orienté signé (i.e. Signed Directed Graph, Puccia & Levins, 1985), dans lequel les nœuds représentent les variables d’états et les lignes représentent les interactions causales entre les variables. Ce type de graphique orienté est transposable en matrice d’adjacence (Levins, 1968), qui permet de regrouper les liens entre les variables en une matrice (Fig. 2). Ici, ces relations sont dites qualitatives car seules les signes des interactions sont spécifiés. Dans cette 

