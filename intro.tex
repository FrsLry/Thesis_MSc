% Ricardo RISO
% \part{Partie 2}

    \chapter{Introduction}
Les structures tridimensionnelles que forment les récifs biogéniques constituent des habitats complexes pour les communautés benthiques. Les espèces ingénieures (Jones et al., 1994, 1997) à l’origine de ces bio-constructions jouent donc un rôle structurant d’un point de vue écosystémique et environnemental. Dans les régions tempérées côtières, certains annélides polychètes de la famille des Sabellariidae possèdent ces qualités de constructeurs de récifs. L’espèce grégaire Sabellaria alveolata (Linnaeus, 1967), plus connue sous le nom d’hermelle, peut former ce type de constructions complexes augmentant localement la richesse spécifique (Lecornu et al. 2016). Cette espèce peut se rencontrer tout le long des côtes ouest européennes (i.e. du nord de l’Angleterre au sud du Portugal) et s’étend jusqu’au sud des côtes marocaines en passant par la Manche (Gruet & Lassus, 1983 à trouver). On la trouve aussi en Méditerranée. Cet annélide sédentaire se fixe sur des substrats rocheux où elle forme différentes structures,allant d’un simple encroûtement à d’imposantes structures récifales (Gruet et al.1982 à trouver archimer ?, ). Le site de Sainte-Anne au sud de la Baie du Mont-Saint-Michel constitue un exemple emblématique de récif d’hermelle consolidé sur les fonds meubles. Sa couverture, d’environ 100 hectares lui vaut d’être qualifié de plus large formation récifale d’Europe (Gruet & Bodeur, 1997). Un effet tampon émerge de ces bioconstructions complexes qui protègent localement l’écosystème en altérant les pressions hydrodynamiques et chimiques des masses d’eau et en fournissant une protection pour de nombreuses espèces benthiques etdémersales. Ces structures sont donc souvent considérées comme des hotspots de biodiversités (Jones et al 2018).  
Etant donnée son importance écologique, la biologie et l’écologie de cette espèce, ainsi que son apport sur la complexité topographique, représentent des points clefs dans la compréhension de la dynamique de ces écosystèmes. Relativement à d’autres espèces ingénieurs, S. alveolata reste malgré tout peu connue, la majorité de la littérature se focalisant sur la relation de cette espèce avec le sédiment et sur l’identification des épibiontes (e.g. Jones, 2017). Les réactions et prédictions des évolutions des communautés accompagnant S. alveolata restent donc à explorer. Dans cette optique de compréhension globale d’un écosystème, la modélisation représente un outil permettant de mieux comprendre et prédire les réponses des récifs et des communautés associées afin d’observer et simuler in silico les répercussions de changements affectant les communautés. Ici, notre travail consiste à modéliser la structure des interactions entre récifs d’hermelle et communautés associées afin de mieux comprendre et prédire la dynamique et leur réponse aux perturbations. La modélisation permet notamment de caractériser la contribution des interactions inter-spécifiques et des fluctuations environnementales à la structure et à la dynamique des communautés associées au récif. Ici, deux types de modèles complémentaires seront utilisées : (1) la modélisation qualitative et (2) les réseaux bayésiens.

	    \section{Le Projet REEHAB}
Ce stage s’inscrit dans le cadre du projet REEHAB (Honeycomb Worm REEf Habitat in Europe) dont l’objectif global est d’aborder la question de l’état de santé des récifs à S. alveolata afin, à terme, d’aider les parties prenantes dans la gestion de ces écosystèmes. Ce programme s’organise en cinq volets : (1) cartographier la distribution des récifs, (2) effectuer un suivit des bio-constructions et de leurs principaux compétiteurs, (3) mesurer l’état de santé des individus et des populations de S. alveolata, (4) développer un indice d’état de santé universel et reproductible de ces bio-constructions et (5) prédire les changements dans la distribution de ces récifs grâce à la modélisation. Ce stage concerne donc ce dernier volet (5). L’outil de modélisation servira ici à déterminer, après validation de la structure du modèle, la conséquence de scénarios futurs (e.g. changement climatique ou stratégies de gestion locale des impacts anthropogéniques) sur la structure des communautés accompagnant les hermelles et sur la probabilité d’occurrence des différents états du récif. 

        \section{Structure du rapport}
Dans un premier temps, la modélisation qualitative (Puccia et Levins, 1985) est utilisée afin d’observer les réponses de différentes topologies de modèles à différents scénarios perturbateurs (e.g. augmentation du substrat rocheux, diminution d’un ou de plusieurs compétiteurs, etc). Cette première étape a pour objectif la délimitation et la caractérisation du modèle le plus représentatif de la complexité observée, à la fois dans sa structure et dans sa dynamique face à différentes perturbations. En revanche, la modélisation qualitative fait l’assertion d’une binarité dans les relations des éléments de nos modèles (i.e. relation soit positive soit négative) et ne prends pas en compte les nuances quantitatives et les seuils régissant ces relations in situ. \hl{C’est pourquoi, dans un second temps,…(Réseaux bayésiens)} 